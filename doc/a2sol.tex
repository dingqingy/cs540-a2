\documentclass{article}

\usepackage{fullpage}
\usepackage{color}
\usepackage{amsmath}
\usepackage{url}
\usepackage{verbatim}
\usepackage{graphicx}
\usepackage{parskip}
\usepackage{amssymb}
\usepackage{nicefrac}
\usepackage{listings} % For displaying code
\usepackage{algorithm2e} % pseudo-code

% Answers
\def\ans#1{\par\gre{Answer: #1}}
%\def\ans#1{} % Comment this line to produce document with answers

% Colors
\definecolor{blu}{rgb}{0,0,1}
\def\blu#1{{\color{blu}#1}}
\definecolor{gre}{rgb}{0,.5,0}
\def\gre#1{{\color{gre}#1}}
\definecolor{red}{rgb}{1,0,0}
\def\red#1{{\color{red}#1}}
\def\norm#1{\|#1\|}

% Math
\def\R{\mathbb{R}}
\def\argmax{\mathop{\rm arg\,max}}
\newcommand{\argmin}[1]{\mathop{\hbox{argmin}}_{#1}}
\newcommand{\mat}[1]{\begin{bmatrix}#1\end{bmatrix}}
\newcommand{\alignStar}[1]{\begin{align*}#1\end{align*}}
\def\half{\frac 1 2}

% LaTeX
\newcommand{\fig}[2]{\includegraphics[width=#1\textwidth]{a2f/#2}}
\newcommand{\centerfig}[2]{\begin{center}\includegraphics[width=#1\textwidth]{a2f/#2}\end{center}}
\def\items#1{\begin{itemize}#1\end{itemize}}
\def\enum#1{\begin{enumerate}#1\end{enumerate}}


\begin{document}

\title{CPSC 540 Assignment 2 (due February 1 at midnight)}
\author{}
\date{}
\maketitle
\vspace{-4em}

The assignment instructions are the same as for the previous assignment, but for this assignment you can work in groups of 1-3. However, please only hand in one assignment for the group.


\blu{\enum{
\item Name(s):
\item Student ID(s):
}}




\section{Calculation Questions}


\subsection{Convexity}

\blu{Show that the following functions are convex, by only using one of the definitions of convexity (i.e., without using the ``operations that preserve convexity" or using convexity results stated in class)}:\footnote{That $C^0$ convex functions are below their chords, that $C^1$ convex functions are above their tangents, or that $C^2$ convex functions have a positive semidefinite Hessian.}
\enum{
\item L2-regularized weighted least squares: $f(w) = \half(Xw - y)^\top V(Xw-y)  + \frac \lambda 2 \norm{w}^2$.\\($V$ is a diagonal matrix with positive values on the diagonal).
\item Poisson regression: $f(w) = -y^\top Xw + 1^\top v$ (where $v_i = \exp(w^\top x^i)$).
\item Weighted infinity-norm: $f(w) = \max_{j \in \{1,2,\dots,d\}}L_j|w_j|$ \red{(where each $L_j \geq 0$)}.\\
Hint: Max and absolute value are not differentiable in general, so you cannot use the Hessian for this question.
}

\blu{Show that the following functions are convex (you can use results from class and operations that preserve convexity if they help)}:
\enum{
\setcounter{enumi}{3}
\item Regularized regression with arbitrary $p$-norm and weighted $q$-norm: $f(w) = \norm{Xw - y}_p + \lambda\norm{Aw}_q$.
\item Support vector regression: $f(w) = \sum_{i=1}^N\max\{0, |w^\top x_i - y_i| - \epsilon\} + \frac{\lambda}{2}\norm{w}_2^2$.
\item Indicator function for linear constraints: $f(w) = \begin{cases}0 & \text{if $Aw \leq b$}\\\infty & \text{otherwise}\end{cases}$.
}



\subsection{Convergence of Gradient Descent}

For these questions it will be helpful to use the ``convexity inequalities'' notes posted on the webpage.

\enum{
\item In class we showed that if $\nabla f$ is $L$-Lipschitz continuous and $f$ is bounded below then with a step-size of $1/L$ gradient descent is guaranteed to have found a $w^k$ with $\norm{\nabla f(w^k)}^2 \leq \epsilon$ after $t = O(1/\epsilon)$ iterations. Suppose that a more-clever algorithm exists which, on iteration $t$, is guaranteed to have found a $w^k$ satisfying $\norm{\nabla f(w^k)}^2 \leq 2L(f(w^0) - f^*)/t^{4/3}$. \blu{How many iterations of this algorithm would we need to find a $w^k$ with $\norm{\nabla f(w^k)}^2 \leq \epsilon$?}
\item In practice we typically don't know $L$. A common strategy in this setting is to start with some small guess $L^0$ that we know is smaller than the true $L$ (usually we take $L^0=1$). On each iteration $k$, we initialize with $L^k = L^{k-1}$ and we check the inequality
\[
f\left(w^k - \frac{1}{L^k}\nabla f(w^k)\right) \leq f(w^k) - \frac{1}{2L^k}\norm{\nabla f(w^k)}^2.
\]
If this is not satisfied, we double $L^k$ and test it again. This continues until we have an $L^k$ satisfying the inequality, and then we take the step. \blu{Show that gradient descent with $\alpha_k = 1/L^k$ defined in this way has a linear convergence rate of
\[
f(w^k) - f(w^*) \leq \left(1 - \frac{\mu}{2L}\right)^k[f(w^0) - f(w^*)],
\]
\red{if $\nabla f$ is $L$-Lipschitz continuousn and $f$ is $\mu$-strongly convex.}\\
} Hint: if a function is $L$-Lipschitz continuous that it is also $L'$-Lipschitz continuous for any $L' \geq L$.
\item Suppose that, in the previous question, we initialized with $L^k = \red{\half}L^{k-1}$. \blu{Describe a setting where this could work much better}.
\item In class we showed that if $\nabla f$ is $L$-Lipschitz continuous and $f$ is strongly-convex, then with a step-size of $\alpha_k = 1/L$ gradient descent has a convergence rate of 
\[
f(w^k) - f(w^*) = O(\rho^k).
\]
\blu{Show that under these assumptions that a convergence rate of $O(\rho^k)$ in terms of the function values implies that the iterations have a convergence rate of
\[
\norm{w^k - w^*} = O(\rho^{k/2}).
\]}
}




\subsection{Beyond Gradient Descent}


\enum{
\item We can write the proximal-gradient update as
\begin{align*}
w^{k+\half} & = w^k - \alpha_k \nabla f(w^k)\\
w^{k+1} &= \argmin{v\in\R^d}\left\{\frac{1}{2}\norm{v -w^{k+\half}}^2 + \alpha_kr(v)\right\}.
\end{align*}
\blu{Show that this is equivalent to setting
\[
w^{k+1} \in \argmin{v\in\R^d} \left\{ f(w^k) + \nabla f(w^k)^\top (v-w^k) + \frac{1}{2\alpha_k}\norm{v-w^k}^2  +r(v)\right\}.
\]
}
\item The ``sum'' version of multi-class SVMs uses an objective of the form
\[
f(W) = \sum_{i=1}^n \sum_{c \neq y^i}[1 - w_{y^i}^\top x^i + w_c^\top x^i]^+ + \frac \lambda 2 \norm{W}_F^2,
\]
where $[\gamma]^+$ sets negative values to zero (and you can use $k$ as the number of classes so the inner loop is over $(k-1)$ elements). \blu{Derive the sub-differential of this objetive}.
\item In some situations it might be hard to accurately compute the elements of the gradient, but we might have access to the sign of the gradient (this can also be useful in distributed settings where communicating one bit for each element of the gradient is cheaper than communicating a floating point number for each gradient element). 
Consider an $f$ that is bounded below and where $\nabla f$ is Lipschitz continuous in the $\infty$-norm, meaning that
\[
f(v) \leq f(u) + \nabla f(u)^\top (v-u) + \frac{L_\infty}{2}\norm{v-u}_\infty^2,
\]
for all $v$ and $w$ and some $L_\infty$. 
For this setting, consider a sign-based gradient descent algorithm of the form
\[
w^{k+1} = w^k - \frac{\norm{\nabla f(w^k)}_1}{L_\infty}\text{sign}(\nabla f(w^k)),
\]
where we define the sign function element-wise as
\[
\text{sign}(w_j) = \begin{cases}+1 & w_j > 0\\0 & w_j =0\\-1 & w_j < 0\end{cases},
\]
\blu{Show that this sign-based gradient descent algorithm finds a $w^k$ satisfying \red{$\norm{\nabla f(w^k)}^2 \leq \epsilon$} after $t = O(1/\epsilon)$ iterations.}
}


\section{Computation Questions}


\subsection{Proximal-Gradient}


If you run the demo \emph{example\_group.jl}, it will load a dataset and fit a multi-class logistic regression (softmax) classifier. This dataset is actually \emph{linearly-separable}, so there exists a set of weights $W$ that can perfectly classify the training data (though it may be difficult to find a $W$ that perfectly classifiers the validation data). However, 90\% of the columns of $X$ are irrelevant. Because of this issue, when you run the demo you find that the training error is $0$ while the test error is something like $0.2980$.

\enum{
\item Write a new function, \emph{logRegSoftmaxL2}, that fits a multi-class logistic regression model with L2-regularization (this only involves modifying the objective function). \blu{Hand in the modified loss function and report the validation error achieved with $\lambda = 10$ (which is the  best value among powers to 10). Also report the number of non-zero parameters in the model and the number of original features that the model uses}.
\item While L2-regularization reduces overfitting a bit, it still uses all the variables even though 90\% of them are irrelevant. In situations like this, L1-regularization may be more suitable. Write a new function, \emph{logRegSoftmaxL1}, that fits a multi-class logistic regression model with L1-regularization. You can use the function \emph{findMinL1}, which minimizes the sum of a differentiable function and an L1-regularization term.  \blu{Report the number of non-zero parameters in the model and the number of original features that the model uses}.
\item L1-regularization achieves sparsity in the \emph{model parameters}, but in this dataset it's actually the \emph{original features} that are irrelevant. We can encourage sparsity in the original features by using \emph{group} L1-regularization. Write a new function, \emph{proxGradGroupL1}, to allow (disjoint) \emph{group} L1-regularization. Use this within a new function, \emph{softmaxClassiferGL1}, to fit a group L1-regularized multi-class logistic regression model (where \emph{rows} of $W$ are grouped together and we use the L2-norm of the groups).  \blu{Hand in both modified  functions (\emph{logRegSoftmaxGL1} and \emph{proxGradGroupL1}) and report the validation error achieved with $\lambda=10$. Also report the number of non-zero parameters in the model and the number of original features that the model uses}.
}


\subsection{Coordinate Optimization}

The function \emph{example\_CD.jl} loads a dataset and tries to fit an L2-regularized least squares model using coordinate descent. Unfortunately, if we use $L_f$ as the Lipschitz constant of $\nabla f$, the runtime of this procedure is $O(d^3 + nd^2\frac{L_f}{\mu}\log(1/\epsilon))$. This comes from spending $O(d^3)$ computing $L_f$, having an iteration cost of $O(nd)$, and requiring $O(d\frac{L_f}{\mu}\log(1/\epsilon))$ iterations to reach an accuracy of $\epsilon$. This non-ideal runtime is also reflected in practice: the algorithm's iterations are relatively slow and it often takes over 200 ``passes'' through the data for the parameters to stabilize.

\enum{
\item Modify this code so that the runtime of the algorithm is $O(nd\frac{L_c}{\mu}\log(1/\epsilon))$, where $L_c$ is the Lipschitz constant of \emph{all} partial derivatives $\nabla_i f$. You can do this by increasing the step-size to $1/L_c$ (the coordinate-wise Lipschitz constant given by $\max_j\{\norm{x_j}^2\} + \lambda$ where $x_j$ is column $j$ of the matrix $X$), and modifying hte iterations so they have a cost of $O(n)$ instead of $O(nd)$.
 \blu{Hand in your code and report an estimate of the change in time and number of iterations}.
 \item While it doesn't improve the worst-case time complexity (without making stronger assumptions), you might expect to improve performance by using a more-clever choice of step-size. \blu{Modify the code to compute the optimal step-size (which you can do in closed-form without increasing the runtime), and report the effect of using the optimal step-size on the time and number of iterations.}
}

\subsection{Stochastic Gradient}

Coming soon...


\section{Very-Short Answer Questions}

Coming soon...



 
\end{document}